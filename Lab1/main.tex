\documentclass[12pt,onecolumn]{article}
\usepackage[utf8]{inputenc} % UTF8 input encoding
\usepackage[T2A]{fontenc}   % T2A font encoding for Cyrillic script
\usepackage[russian]{babel} % Russian language support
\usepackage{listings}
\usepackage{float}
\usepackage{mathtools}
\usepackage{longtable}
\everymath{\displaystyle}
\usepackage{listings} 
\usepackage[usenames]{color}
\usepackage[html]{xcolor}
\usepackage{framed}
\usepackage{csquotes}
\usepackage{geometry}

\geometry{
  a4paper,
  top=25mm, 
  right=5mm, 
  bottom=25mm, 
  left=5mm
}

\definecolor{mygreen}{rgb}{0,0.6,0}
\definecolor{mygray}{rgb}{0.5,0.5,0.5}
\definecolor{mymauve}{rgb}{0.58,0,0.82}
\definecolor{terminalbgcolor}{HTML}{330033}
\definecolor{terminalrulecolor}{HTML}{000099}
\newcommand{\lstconsolestyle}{
\lstset{
	basicstyle=\color{black}\linespread{1.15}\fontfamily{fvm}\scriptsize\selectfont,
	breakatwhitespace=false,  
	breaklines=true,
	captionpos=b,
	commentstyle=\color{mygreen},
	deletekeywords={...},
	escapeinside={\%*}{*)},
	extendedchars=true,
	frame=single,
	keepspaces=true,
	keywordstyle=\color{blue},
	%language=none,
	morekeywords={*,...},
	numbers=left,
	numbersep=5pt,
  framerule=1pt,
	numberstyle=\color{mygray}\tiny\selectfont,
	rulecolor=\color{terminalbgcolor},
	showspaces=false,
	showstringspaces=false,
	showtabs=false,
	stepnumber=2,
	stringstyle=\color{mymauve},
	tabsize=2,
  literate={а}{{\selectfont\char224}}1
  {б}{{\selectfont\char225}}1
  {в}{{\selectfont\char226}}1
  {г}{{\selectfont\char227}}1
  {д}{{\selectfont\char228}}1
  {е}{{\selectfont\char229}}1
  {ё}{{\"e}}1
  {ж}{{\selectfont\char230}}1
  {з}{{\selectfont\char231}}1
  {и}{{\selectfont\char232}}1
  {й}{{\selectfont\char233}}1
  {к}{{\selectfont\char234}}1
  {л}{{\selectfont\char235}}1
  {м}{{\selectfont\char236}}1
  {н}{{\selectfont\char237}}1
  {о}{{\selectfont\char238}}1
  {п}{{\selectfont\char239}}1
  {р}{{\selectfont\char240}}1
  {с}{{\selectfont\char241}}1
  {т}{{\selectfont\char242}}1
  {у}{{\selectfont\char243}}1
  {ф}{{\selectfont\char244}}1
  {х}{{\selectfont\char245}}1
  {ц}{{\selectfont\char246}}1
  {ч}{{\selectfont\char247}}1
  {ш}{{\selectfont\char248}}1
  {щ}{{\selectfont\char249}}1
  {ъ}{{\selectfont\char250}}1
  {ы}{{\selectfont\char251}}1
  {ь}{{\selectfont\char252}}1
  {э}{{\selectfont\char253}}1
  {ю}{{\selectfont\char254}}1
  {я}{{\selectfont\char255}}1
  {А}{{\selectfont\char192}}1
  {Б}{{\selectfont\char193}}1
  {В}{{\selectfont\char194}}1
  {Г}{{\selectfont\char195}}1
  {Д}{{\selectfont\char196}}1
  {Е}{{\selectfont\char197}}1
  {Ё}{{\"E}}1
  {Ж}{{\selectfont\char198}}1
  {З}{{\selectfont\char199}}1
  {И}{{\selectfont\char200}}1
  {Й}{{\selectfont\char201}}1
  {К}{{\selectfont\char202}}1
  {Л}{{\selectfont\char203}}1
  {М}{{\selectfont\char204}}1
  {Н}{{\selectfont\char205}}1
  {О}{{\selectfont\char206}}1
  {П}{{\selectfont\char207}}1
  {Р}{{\selectfont\char208}}1
  {С}{{\selectfont\char209}}1
  {Т}{{\selectfont\char210}}1
  {У}{{\selectfont\char211}}1
  {Ф}{{\selectfont\char212}}1
  {Х}{{\selectfont\char213}}1
  {Ц}{{\selectfont\char214}}1
  {Ч}{{\selectfont\char215}}1
  {Ш}{{\selectfont\char216}}1
  {Щ}{{\selectfont\char217}}1
  {Ъ}{{\selectfont\char218}}1
  {Ы}{{\selectfont\char219}}1
  {Ь}{{\selectfont\char220}}1
  {Э}{{\selectfont\char221}}1
  {Ю}{{\selectfont\char222}}1
  {Я}{{\selectfont\char223}}1
}
}

\lstdefinestyle{bash}{language=bash, 
  basicstyle=\small\ttfamily,
  commentstyle=\color{cyan},
  stringstyle=\color{magenta}\ttfamily,
  keywordstyle=\color{blue},
  numbers=left,
  numberstyle=\scriptsize,
  numbersep=5pt,
  frame=single,
  breaklines=true,
  breakatwhitespace=true,
  showstringspaces=false,
  tabsize=4,
  inputencoding=utf8,
  extendedchars=true,
  literate={а}{{\selectfont\char224}}1
          {б}{{\selectfont\char225}}1
          {в}{{\selectfont\char226}}1
          {г}{{\selectfont\char227}}1
          {д}{{\selectfont\char228}}1
          {е}{{\selectfont\char229}}1
          {ё}{{\"e}}1
          {ж}{{\selectfont\char230}}1
          {з}{{\selectfont\char231}}1
          {и}{{\selectfont\char232}}1
          {й}{{\selectfont\char233}}1
          {к}{{\selectfont\char234}}1
          {л}{{\selectfont\char235}}1
          {м}{{\selectfont\char236}}1
          {н}{{\selectfont\char237}}1
          {о}{{\selectfont\char238}}1
          {п}{{\selectfont\char239}}1
          {р}{{\selectfont\char240}}1
          {с}{{\selectfont\char241}}1
          {т}{{\selectfont\char242}}1
          {у}{{\selectfont\char243}}1
          {ф}{{\selectfont\char244}}1
          {х}{{\selectfont\char245}}1
          {ц}{{\selectfont\char246}}1
          {ч}{{\selectfont\char247}}1
          {ш}{{\selectfont\char248}}1
          {щ}{{\selectfont\char249}}1
          {ъ}{{\selectfont\char250}}1
          {ы}{{\selectfont\char251}}1
          {ь}{{\selectfont\char252}}1
          {э}{{\selectfont\char253}}1
          {ю}{{\selectfont\char254}}1
          {я}{{\selectfont\char255}}1
          {А}{{\selectfont\char192}}1
          {Б}{{\selectfont\char193}}1
          {В}{{\selectfont\char194}}1
          {Г}{{\selectfont\char195}}1
          {Д}{{\selectfont\char196}}1
          {Е}{{\selectfont\char197}}1
          {Ё}{{\"E}}1
          {Ж}{{\selectfont\char198}}1
          {З}{{\selectfont\char199}}1
          {И}{{\selectfont\char200}}1
          {Й}{{\selectfont\char201}}1
          {К}{{\selectfont\char202}}1
          {Л}{{\selectfont\char203}}1
          {М}{{\selectfont\char204}}1
          {Н}{{\selectfont\char205}}1
          {О}{{\selectfont\char206}}1
          {П}{{\selectfont\char207}}1
          {Р}{{\selectfont\char208}}1
          {С}{{\selectfont\char209}}1
          {Т}{{\selectfont\char210}}1
          {У}{{\selectfont\char211}}1
          {Ф}{{\selectfont\char212}}1
          {Х}{{\selectfont\char213}}1
          {Ц}{{\selectfont\char214}}1
          {Ч}{{\selectfont\char215}}1
          {Ш}{{\selectfont\char216}}1
          {Щ}{{\selectfont\char217}}1
          {Ъ}{{\selectfont\char218}}1
          {Ы}{{\selectfont\char219}}1
          {Ь}{{\selectfont\char220}}1
          {Э}{{\selectfont\char221}}1
          {Ю}{{\selectfont\char222}}1
          {Я}{{\selectfont\char223}}1
}

\newcommand{\sspace}{\hspace{8pt}}

\newcommand{\nquote}[2]{
  \begin{quote}
    \textbf{#1:}\textit{ #2}
  \end{quote}
}


\begin{document}
\setcounter{tocdepth}{4}
\begin{center}
    Федеральное государственное автономное образовательное учреждение высшего образования "Национальный Исследовательский Университет ИТМО"\\ 
    Мегафакультет Компьютерных Технологий и Управления\\
    Факультет Программной Инженерии и Компьютерной Техники \\
    \includegraphics[scale=0.3]{image/itmo.jpg} % нужно закинуть картинку логтипа в папку с отчетом
\end{center}
\vspace{1cm}


\begin{center}
    \textbf{Лабораторная работа №1}\\
    по дисциплине\\
    \textbf{'Операционные системы'}\\
\end{center}

\vspace{2cm}

\begin{flushright}
  Выполнил Студент  группы P32101\\
  \textbf{Лапин Алексей Александрович}\\
  Преподаватель: \\
  \textbf{Осипов Святослав Владимирович}\\
\end{flushright}

\vspace{6cm}
\begin{center}
    г. Санкт-Петербург\\
    2023г.
\end{center}

\newpage
\tableofcontents
\newpage

\section{CPU Testing}
\textbf{Выданные параметры:} [fft,cdouble]\\
\textbf{Параметры процессора используемого в тесте:}\\
{\large Intel Core i5-8259U}\\
Ядер: 4\\
Потоков: 8\\
Базовая частота: 	2.30 ГГц\\
Кэш 1-го уровня: 	64K (на ядро)\\
Кэш 2-го уровня: 	256K (на ядро)\\
Кэш 3-го уровня: 	6 Мб (всего)\\
Входе теста используется ОС Linux установленная на виртуальную машину через Parallels.\\
Ограничение по процессорам: 8\\
Ограничение по памяти: 2048 MB\\
\subsection{fft}
\subsubsection{Команда lscpu}
\begin{verbatim}
  Architecture:            x86_64
  CPU op-mode(s):        32-bit, 64-bit
  Address sizes:         36 bits physical, 48 bits virtual
  Byte Order:            Little Endian
CPU(s):                  8
  On-line CPU(s) list:   0-7
Vendor ID:               GenuineIntel
  Model name:            Intel(R) Core(TM) i5-8259U CPU @ 2.30GHz
    CPU family:          6
    Model:               142
    Thread(s) per core:  1
    Core(s) per socket:  8
    Socket(s):           1
    Stepping:            10
    BogoMIPS:            4608.00
    Flags:               fpu vme de pse tsc msr pae mce cx8 apic sep mtrr pge mc
                         a cmov pat pse36 clflush mmx fxsr sse sse2 ss ht syscal
                         l nx rdtscp lm constant_tsc nopl xtopology nonstop_tsc 
                         cpuid tsc_known_freq pni pclmulqdq ssse3 fma cx16 pcid 
                         sse4_1 sse4_2 x2apic movbe popcnt tsc_deadline_timer ae
                         s xsave avx f16c rdrand hypervisor lahf_lm abm 3dnowpre
                         fetch invpcid_single pti fsgsbase tsc_adjust bmi1 avx2 
                         smep bmi2 invpcid rdseed adx smap clflushopt xsaveopt x
                         savec dtherm arat pln pts
Virtualization features: 
  Hypervisor vendor:     KVM
  Virtualization type:   full
Caches (sum of all):     
  L1d:                   256 KiB (8 instances)
  L1i:                   256 KiB (8 instances)
  L2:                    2 MiB (8 instances)
  L3:                    6 MiB (1 instance)
NUMA:                    
  NUMA node(s):          1
  NUMA node0 CPU(s):     0-7
Vulnerabilities:         
  Gather data sampling:  Unknown: Dependent on hypervisor status
  Itlb multihit:         KVM: Mitigation: VMX unsupported
  L1tf:                  Mitigation; PTE Inversion
  Mds:                   Vulnerable: Clear CPU buffers attempted, no microcode; 
                         SMT Host state unknown
  Meltdown:              Mitigation; PTI
  Mmio stale data:       Vulnerable: Clear CPU buffers attempted, no microcode; 
                         SMT Host state unknown
  Retbleed:              Vulnerable
  Spec rstack overflow:  Not affected
  Spec store bypass:     Vulnerable
  Spectre v1:            Mitigation; usercopy/swapgs barriers and __user pointer
                          sanitization
  Spectre v2:            Mitigation; Retpolines, STIBP disabled, RSB filling, PB
                         RSB-eIBRS Not affected
  Srbds:                 Unknown: Dependent on hypervisor status
  Tsx async abort:       Not affected

\end{verbatim}
Видим, что Linux распознает наши 4 ядра по 2 потока, как 8 ядер по 1 потоку.
\subsubsection{Команда lstopo}
С помощью команды lstopo мы можем посмотреть на топологию нашего процессора:\\
\includegraphics[width=\textwidth]{image/lstopo.png}
\subsubsection{Команда pidstat}
Посмосмотрим статистику по процессу\\
Для этого запустим команду: \textit{stress-ng --cpu 1 --cpu-method fft --metrics --timeout 120}\\
И посмотрим статистику запустив команду: \textit{pidstat -p 58452 1}, где pid мы узнали спомощью команды \textit{top}.
\begin{verbatim}
20:45:33      UID       PID    %usr %system  %guest   %wait    %CPU   CPU  Command
20:45:34     1000     58452  100,00    0,00    0,00    0,00  100,00     3  stress-ng
20:45:35     1000     58452  100,00    0,00    0,00    0,00  100,00     3  stress-ng
20:45:36     1000     58452  100,00    0,00    0,00    0,00  100,00     3  stress-ng
20:45:37     1000     58452  100,00    0,00    0,00    0,00  100,00     3  stress-ng
20:45:38     1000     58452  100,00    0,00    0,00    0,00  100,00     3  stress-ng
20:45:39     1000     58452   99,01    0,00    0,00    0,00   99,01     3  stress-ng
20:45:40     1000     58452  100,00    0,00    0,00    0,00  100,00     3  stress-ng
20:45:41     1000     58452  101,00    0,00    0,00    0,00  101,00     3  stress-ng
20:45:42     1000     58452  100,00    0,00    0,00    0,00  100,00     3  stress-ng
20:45:43     1000     58452  100,00    0,00    0,00    0,00  100,00     3  stress-ng
20:45:44     1000     58452   99,01    0,00    0,00    0,00   99,01     3  stress-ng
20:45:45     1000     58452  100,00    0,00    0,00    0,00  100,00     3  stress-ng
20:45:46     1000     58452  100,00    0,00    0,00    0,00  100,00     3  stress-ng
20:45:47     1000     58452  100,00    0,00    0,00    0,00  100,00     3  stress-ng
20:45:48     1000     58452  100,00    0,00    0,00    0,00  100,00     3  stress-ng
20:45:49     1000     58452  100,00    0,00    0,00    0,00  100,00     3  stress-ng
20:45:50     1000     58452  100,00    0,00    0,00    0,00  100,00     3  stress-ng
20:45:51     1000     58452  100,00    0,00    0,00    0,00  100,00     3  stress-ng
20:45:52     1000     58452  100,00    0,00    0,00    0,00  100,00     3  stress-ng
20:45:53     1000     58452  101,00    0,00    0,00    0,00  101,00     3  stress-ng
20:45:54     1000     58452  100,00    0,00    0,00    0,00  100,00     3  stress-ng
20:45:55     1000     58452  100,00    0,00    0,00    0,00  100,00     3  stress-ng
20:45:56     1000     58452   99,01    0,00    0,00    0,00   99,01     3  stress-ng
20:45:57     1000     58452  100,00    0,00    0,00    0,00  100,00     3  stress-ng
20:45:58     1000     58452  100,00    0,00    0,00    0,00  100,00     3  stress-ng
20:45:59     1000     58452  100,00    0,00    0,00    0,00  100,00     3  stress-ng
20:46:00     1000     58452  100,00    0,00    0,00    0,00  100,00     3  stress-ng
20:46:01     1000     58452  100,00    0,00    0,00    0,00  100,00     3  stress-ng
20:46:02     1000     58452  100,00    0,00    0,00    0,00  100,00     3  stress-ng
20:46:03     1000     58452  100,00    0,00    0,00    0,00  100,00     3  stress-ng
^C
Average:     1000     58452   99,97    0,00    0,00    0,00   99,97     -  stress-ng
\end{verbatim}
Мы видим, что процесс выполнялся на процессоре номер 3 и занимал почти все его время(100\%).
При этом процесс выполнялся практически все время в режиме \textit{пользователя}, что объяснимо так как fft - вычислительная задача.\\
Результат \textit{stress-ng}:
\lstconsolestyle
\definecolor{xxxhtmlcolorIMAIMP}{HTML}{26A269}
\definecolor{xxxhtmlcolorHIKOOB}{HTML}{12488B}
\begin{lstlisting}
%*{\color{xxxhtmlcolorIMAIMP}
  {\bfseries alexey@alexey{-}ubuntu}}*):%*{\color{xxxhtmlcolorHIKOOB}{\bfseries \textasciitilde{}}}*)$ stress-ng --cpu 1 --cpu-method fft --metrics --timeout 120

stress-ng: info:  [58451] setting to a 120 second (2 mins, 0.00 secs) run per stressor

stress-ng: info:  [58451] dispatching hogs: 1 cpu

stress-ng: info:  [58451] successful run completed in 120.00s (2 mins, 0.00 secs)

stress-ng: info:  [58451] stressor  bogo ops real time  usr time  sys time   bogo ops/s     bogo ops/s CPU used per

stress-ng: info:  [58451]                    (secs)    (secs)    (secs)   (real time) (usr+sys time) instance (%)

stress-ng: info:  [58451] cpu       102216    120.00    119.97      0.00       851.79         852.01        99.97
\end{lstlisting}

Также посмотрим на приоритет и политику планирования процесса: \\
{\lstconsolestyle
\definecolor{xxxhtmlcolorIMAIMP}{HTML}{26A269}
\definecolor{xxxhtmlcolorHIKOOB}{HTML}{12488B}\begin{lstlisting}
%*{\color{xxxhtmlcolorIMAIMP}{\bfseries alexey@alexey{-}ubuntu}}*):%*{\color{xxxhtmlcolorHIKOOB}{\bfseries \textasciitilde{}}}*)$ pidstat -R -p 76043 1

Linux 6.2.0-35-generic (alexey-ubuntu) 	22.10.2023 	_x86_64_	(8 CPU)



22:28:20      UID       PID     prio policy  Command

22:28:21   %*{\color{xxxhtmlcolorIMAIMP}\sspace \sspace 1000\sspace \sspace \sspace \sspace \sspace 76043}*)%*{\color{xxxhtmlcolorHIKOOB}\sspace \sspace \sspace \sspace 0}*)%*{\color{xxxhtmlcolorHIKOOB}{\bfseries \sspace NORMAL\sspace \sspace stress{-}ng}}*)

22:28:22   %*{\color{xxxhtmlcolorIMAIMP}\sspace \sspace 1000\sspace \sspace \sspace \sspace \sspace 76043}*)%*{\color{xxxhtmlcolorHIKOOB}\sspace \sspace \sspace \sspace 0}*)%*{\color{xxxhtmlcolorHIKOOB}{\bfseries \sspace NORMAL\sspace \sspace stress{-}ng}}*)

22:28:23   %*{\color{xxxhtmlcolorIMAIMP}\sspace \sspace 1000\sspace \sspace \sspace \sspace \sspace 76043}*)%*{\color{xxxhtmlcolorHIKOOB}\sspace \sspace \sspace \sspace 0}*)%*{\color{xxxhtmlcolorHIKOOB}{\bfseries \sspace NORMAL\sspace \sspace stress{-}ng}}*)

22:28:24   %*{\color{xxxhtmlcolorIMAIMP}\sspace \sspace 1000\sspace \sspace \sspace \sspace \sspace 76043}*)%*{\color{xxxhtmlcolorHIKOOB}\sspace \sspace \sspace \sspace 0}*)%*{\color{xxxhtmlcolorHIKOOB}{\bfseries \sspace NORMAL\sspace \sspace stress{-}ng}}*)

22:28:25   %*{\color{xxxhtmlcolorIMAIMP}\sspace \sspace 1000\sspace \sspace \sspace \sspace \sspace 76043}*)%*{\color{xxxhtmlcolorHIKOOB}\sspace \sspace \sspace \sspace 0}*)%*{\color{xxxhtmlcolorHIKOOB}{\bfseries \sspace NORMAL\sspace \sspace stress{-}ng}}*)

22:28:26   %*{\color{xxxhtmlcolorIMAIMP}\sspace \sspace 1000\sspace \sspace \sspace \sspace \sspace 76043}*)%*{\color{xxxhtmlcolorHIKOOB}\sspace \sspace \sspace \sspace 0}*)%*{\color{xxxhtmlcolorHIKOOB}{\bfseries \sspace NORMAL\sspace \sspace stress{-}ng}}*)

22:28:27   %*{\color{xxxhtmlcolorIMAIMP}\sspace \sspace 1000\sspace \sspace \sspace \sspace \sspace 76043}*)%*{\color{xxxhtmlcolorHIKOOB}\sspace \sspace \sspace \sspace 0}*)%*{\color{xxxhtmlcolorHIKOOB}{\bfseries \sspace NORMAL\sspace \sspace stress{-}ng}}*)

22:28:28   %*{\color{xxxhtmlcolorIMAIMP}\sspace \sspace 1000\sspace \sspace \sspace \sspace \sspace 76043}*)%*{\color{xxxhtmlcolorHIKOOB}\sspace \sspace \sspace \sspace 0}*)%*{\color{xxxhtmlcolorHIKOOB}{\bfseries \sspace NORMAL\sspace \sspace stress{-}ng}}*)

22:28:29   %*{\color{xxxhtmlcolorIMAIMP}\sspace \sspace 1000\sspace \sspace \sspace \sspace \sspace 76043}*)%*{\color{xxxhtmlcolorHIKOOB}\sspace \sspace \sspace \sspace 0}*)%*{\color{xxxhtmlcolorHIKOOB}{\bfseries \sspace NORMAL\sspace \sspace stress{-}ng}}*)

^C

Average:   %*{\color{xxxhtmlcolorIMAIMP}\sspace \sspace 1000\sspace \sspace \sspace \sspace \sspace 76043}*)%*{\color{xxxhtmlcolorHIKOOB}\sspace \sspace \sspace \sspace 0}*)%*{\color{xxxhtmlcolorHIKOOB}{\bfseries \sspace NORMAL\sspace \sspace stress{-}ng}}*)
\end{lstlisting}}
\begin{quote}
  \textit{
  \textbf{NORMAL:} SCHED\_NORMAL (ранее известный как SCHED\_OTHER) — это алгоритм планирования с разделением времени. Используется по умолчанию для пользовательских процессов. Планировщик динамически корректирует приоритет в зависимости от класса планирования. Для $O(1)$ длительность кванта времени устанавливается на основе статического приоритета: чем выше приоритет, тем больше квант времени. Для класса CFS размер кванта выбирается динамически. Использует класс планирования CFS.
  }
\end{quote}

\subsubsection{Команда mpstat}
Запустим команду \textit{mpstat}, чтобы посмотреть загруженность всех процессоров.\\
\includegraphics[width=\textwidth]{image/mpstat.png}
Видим, что \textit{stress-ng} загружает только один процессор на 100\% (в данном случае процессор 4).
Такое поведение и ожидалось так как мы запустили \textit{stress-ng} с параматром \textit{--cpu 1}.
Для следующих запусков определим командой \textit{taskset} процессор на котором будет выполняться \textit{stress-ng}.\\
\includegraphics[width=\textwidth]{image/taskset.png}
Как мы можем заметить, производительность немного подросла, особенно это видно при повторном запуске. Это следует из того, что при использовании одного и того же процессора кеши остаются горячими.\\ 
\includegraphics[width=\textwidth]{image/mpstat2.png}
\subsubsection{Команда atop}
С помощью команды \textit{atop}, которая является более специфичным для Linux аналогом команды top, мы посмотрим общую загруженность системы.\\
\includegraphics[width=\textwidth]{image/atop2.png}
Отсюда мы делаем вывод, что кроме нашего процесса самое высокое потребление у \textit{gnome-shell}, но пока мы рассматриваем нагрузку только на одно ядро для нас это неважно, потому что gnome-shell будет выполняться на другом ядре.\\
\includegraphics[width=\textwidth]{image/atop3.png}
Запустили без графической оболочки. \\
Как и ожидалось прироста производительности нет, пока мы рассматриваем тесты для одного ядра.\\
\includegraphics[width=\textwidth]{image/gnome.png}
\subsubsection{Команда stress-ng}
Воспользуемся встроенными параметрами \textit{stress-ng}.\\
С помощью опции --times можно получить представление о том, сколько пользовательского и системного (ядро) времени используется.\\
Мы можем запустить \textit{stress-ng} с опцией --perf, чтобы увидеть более подробную информацию о том, что делает машина во время работы\\
{\lstconsolestyle
\definecolor{xxxhtmlcolorIMAIMP}{HTML}{26A269}
\definecolor{xxxhtmlcolorHIKOOB}{HTML}{12488B}\begin{lstlisting}
%*{\color{xxxhtmlcolorIMAIMP}{\bfseries alexey@alexey{-}ubuntu}}*):%*{\color{xxxhtmlcolorHIKOOB}{\bfseries \textasciitilde{}}}*)$ taskset -c 0 sudo stress-ng --cpu 1 --cpu-method fft --metrics --timeout 120 --times --perf
[sudo] password for alexey: 
stress-ng: info:  [3347] setting to a 120 second (2 mins, 0.00 secs) run per stressor
stress-ng: info:  [3347] dispatching hogs: 1 cpu
stress-ng: info:  [3347] successful run completed in 121.46s (2 mins, 1.46 secs)
stress-ng: info:  [3347] stressor       bogo ops real time  usr time  sys time   bogo ops/s     bogo ops/s CPU used per
stress-ng: info:  [3347]                           (secs)    (secs)    (secs)   (real time) (usr+sys time) instance (%)
stress-ng: info:  [3347] cpu              109925    121.46    119.88      0.05       905.06         916.58        98.74
stress-ng: info:  [3347] cpu:
stress-ng: info:  [3347]      119 873 637 761 CPU Clock                       0.99 B/sec
stress-ng: info:  [3347]      119 867 619 227 Task Clock                      0.99 B/sec
stress-ng: info:  [3347]                         44 Page Faults Total               0,36 /sec 
stress-ng: info:  [3347]                         44 Page Faults Minor               0,36 /sec 
stress-ng: info:  [3347]                          0 Page Faults Major               0,00 /sec 
stress-ng: info:  [3347]                        763 Context Switches                6,28 /sec 
stress-ng: info:  [3347]                        761 Cgroup Switches                 6,27 /sec 
stress-ng: info:  [3347]                          0 CPU Migrations                  0,00 /sec 
stress-ng: info:  [3347]                          0 Alignment Faults                0,00 /sec 
stress-ng: info:  [3347]                          0 Emulation Faults                0,00 /sec 
stress-ng: info:  [3347]                         43 Page Faults User                0,35 /sec 
stress-ng: info:  [3347]                          1 Page Faults Kernel              0,01 /sec 
stress-ng: info:  [3347]                         92 System Call Enter               0,76 /sec 
stress-ng: info:  [3347]                         91 System Call Exit                0,75 /sec 
stress-ng: info:  [3347]                          1 Kmalloc                         0,01 /sec 
stress-ng: info:  [3347]                          1 Kfree                           0,01 /sec 
stress-ng: info:  [3347]                          3 Kmem Cache Alloc                0,02 /sec 
stress-ng: info:  [3347]                          2 Kmem Cache Free                 0,02 /sec 
stress-ng: info:  [3347]                         35 MM Page Alloc                   0,29 /sec 
stress-ng: info:  [3347]                          0 MM Page Free                    0,00 /sec 
stress-ng: info:  [3347]                   79.202 RCU Utilization               652.10 /sec 
stress-ng: info:  [3347]                          9 Sched Migrate Task              0,07 /sec 
stress-ng: info:  [3347]                          0 Sched Move NUMA                 0,00 /sec 
stress-ng: info:  [3347]                        765 Sched Wakeup                    6,30 /sec 
stress-ng: info:  [3347]                          0 Sched Proc Exec                 0,00 /sec 
stress-ng: info:  [3347]                          0 Sched Proc Exit                 0,00 /sec 
stress-ng: info:  [3347]                          0 Sched Proc Fork                 0,00 /sec 
stress-ng: info:  [3347]                          0 Sched Proc Free                 0,00 /sec 
stress-ng: info:  [3347]                          0 Sched Proc Hang                 0,00 /sec 
stress-ng: info:  [3347]                          0 Sched Proc Wait                 0,00 /sec 
stress-ng: info:  [3347]                        763 Sched Switch                    6,28 /sec 
stress-ng: info:  [3347]                          2 Signal Generate                 0,02 /sec 
stress-ng: info:  [3347]                          1 Signal Deliver                  0,01 /sec 
stress-ng: info:  [3347]                         32 IRQ Entry                       0,26 /sec 
stress-ng: info:  [3347]                         32 IRQ Exit                        0,26 /sec 
stress-ng: info:  [3347]                   11.185 Soft IRQ Entry                 92.09 /sec 
stress-ng: info:  [3347]                   11.185 Soft IRQ Exit                  92.09 /sec 
stress-ng: info:  [3347]                          0 NMI handler                     0,00 /sec 
stress-ng: info:  [3347]                          1 Writeback Dirty Inode           0,01 /sec 
stress-ng: info:  [3347]                          0 Migrate MM Pages                0,00 /sec 
stress-ng: info:  [3347]                          1 SKB Consume                     0,01 /sec 
stress-ng: info:  [3347]                          0 SKB Kfree                       0,00 /sec 
stress-ng: info:  [3347]                          0 IOMMU IO Page Fault             0,00 /sec 
stress-ng: info:  [3347]                          0 IOMMU Map                       0,00 /sec 
stress-ng: info:  [3347]                          0 IOMMU Unmap                     0,00 /sec 
stress-ng: info:  [3347]                          0 Filemap page-cache add          0,00 /sec 
stress-ng: info:  [3347]                          0 Filemap page-cache del          0,00 /sec 
stress-ng: info:  [3347]                          0 OOM Compact Retry               0,00 /sec 
stress-ng: info:  [3347]                          0 OOM Wake Reaper                 0,00 /sec 
stress-ng: info:  [3347]                          0 Thermal Zone Trip               0,00 /sec 
stress-ng: info:  [3347] for a 121,46s run time:
stress-ng: info:  [3347]     971,66s available CPU time
stress-ng: info:  [3347]     119,88s user time   ( 12,34%)
stress-ng: info:  [3347]       0,05s system time (  0,01%)
stress-ng: info:  [3347]     119,93s total time  ( 12,34%)
stress-ng: info:  [3347] load average: 1,09 0,69 0,30
\end{lstlisting}}
\subsubsection{Flame Graph}
Выполняем профилирование в течении 10 секунд.\\
\includegraphics[width=\textwidth]{image/perf.png}
Строим по выводу Flame Graph.\\
\includegraphics[width=\textwidth]{image/flamegraph.png}
\subsection{Переходим к активным действиям: nice}
Повысим приоритет нашего процесса с помощью команды \textit{nice}.\\
Чтобы была конкуренция за ресурс процессора, мы запустим тест на все ядра (8 ядер)\\
\includegraphics[width=\textwidth]{image/nice.png}
Видим, что с увеличением приоритета мы немного выиграли в производительности.
\subsubsection{Команда chrt}
Попробуем разные политики планирования для наших процессов.\\
\nquote{RR (Round Robin)}{SCHED\_RR — это алгоритм планирования циклическим перебором. Как только поток исчерпает свой квант времени, он перемещается в конец очереди на выполнение для своего уровня приоритета. Это позволяет запускать другие потоки с таким же приоритетом. Использует класс планирования RT.}
\includegraphics[width=\textwidth]{image/rr.png}
\nquote{FIFO (First In, First Out)}{SCHED\_FIFO — это алгоритм планирования «первым пришел, первым вышел», который продолжает выполнять поток, находящийся в голове очереди на выполнение, пока тот не оставит процессор добровольно или не появится поток с более высоким приоритетом. Поток продолжает выполняться, даже если в очереди на выполнение есть другие потоки с таким же приоритетом. Использует класс планирования RT.}
\includegraphics[width=\textwidth]{image/fifo.png}
\nquote{NORMAL}{SCHED\_NORMAL (ранее известный как SCHED\_OTHER) — это алгоритм планирования с разделением времени. Используется по умолчанию для пользовательских процессов. Планировщик динамически корректирует приоритет в зависимости от класса планирования. Для O(1) длительность кванта времени устанавливается на основе статического приоритета: чем выше приоритет, тем больше квант времени. Для класса CFS размер кванта выбирается динамически. Использует класс планирования CFS.}
\includegraphics[width=\textwidth]{image/normal.png}
\nquote{BATCH}{SCHED\_BATCH — этот алгоритм действует как SCHED\_NORMAL, но ожидается, что поток будет привязан к процессору и не должен прерывать другие интерактивные задачи, связанные с вводом/выводом. Использует класс планирования CFS.}
\includegraphics[width=\textwidth]{image/batch.png}
\nquote{IDLE:}{SCHED\_IDLE использует класс планирования Idle.}
\includegraphics[width=\textwidth]{image/idle.png}
Как мы видим, выбор политики планирования не сильно влияет на производительность нашего теста.
\subsubsection{Финал}
Отключаем графический интерфейс и применяем все настройки выше. \\
\includegraphics[width=\textwidth]{image/cpu-max.png}
Итого мы с \textbf{417426} bogo ops смогли повысить проиизводительность до \textbf{496438} bogo ops.
\subsection{cdouble}
Так как \textit{cdouble} тоже вычислительная задача, то попробуем применить все настройки выше.\\
Но сначала тест без настроек. \\
\includegraphics[width=\textwidth]{image/cdouble-1.png}
Посмотрим потребление процессора.\\
\begin{verbatim}
14:39:30      UID       PID    %usr %system  %guest   %wait    %CPU   CPU  Command
14:39:31     1000     16167   96,00    1,00    0,00    4,00   97,00     0  stress-ng
14:39:32     1000     16167   99,01    0,00    0,00    0,99   99,01     0  stress-ng
14:39:33     1000     16167  100,00    0,00    0,00    0,00  100,00     0  stress-ng
14:39:34     1000     16167   96,00    0,00    0,00    2,00   96,00     0  stress-ng
14:39:35     1000     16167   96,00    1,00    0,00    4,00   97,00     0  stress-ng
14:39:36     1000     16167   97,00    1,00    0,00    3,00   98,00     0  stress-ng
14:39:37     1000     16167   95,00    0,00    0,00    4,00   95,00     0  stress-ng
14:39:38     1000     16167   99,00    1,00    0,00    1,00  100,00     0  stress-ng
14:39:39     1000     16167   95,00    1,00    0,00    4,00   96,00     5  stress-ng
14:39:40     1000     16167   95,00    0,00    0,00    5,00   95,00     5  stress-ng
14:39:41     1000     16167   97,00    0,00    0,00    3,00   97,00     5  stress-ng
14:39:42     1000     16167   96,00    0,00    0,00    3,00   96,00     5  stress-ng
14:39:43     1000     16167   97,00    1,00    0,00    3,00   98,00     5  stress-ng
14:39:44     1000     16167   96,00    1,00    0,00    3,00   97,00     5  stress-ng
14:39:45     1000     16167   95,00    1,00    0,00    4,00   96,00     5  stress-ng
14:39:46     1000     16167   93,00    1,00    0,00    6,00   94,00     5  stress-ng
14:39:47     1000     16167   95,05    0,99    0,00    2,97   96,04     5  stress-ng
14:39:48     1000     16167   98,00    0,00    0,00    3,00   98,00     5  stress-ng
14:39:49     1000     16167   95,05    1,98    0,00    1,98   97,03     5  stress-ng
14:39:50     1000     16167   95,00    2,00    0,00    3,00   97,00     3  stress-ng
14:39:51     1000     16167   97,00    0,00    0,00    3,00   97,00     3  stress-ng
14:39:52     1000     16167   95,05    0,99    0,00    3,96   96,04     3  stress-ng
14:39:53     1000     16167   96,00    0,00    0,00    3,00   96,00     3  stress-ng
14:39:54     1000     16167   97,00    1,00    0,00    4,00   98,00     3  stress-ng
14:39:55     1000     16167   95,00    0,00    0,00    3,00   95,00     3  stress-ng
14:39:56     1000     16167   95,00    2,00    0,00    5,00   97,00     3  stress-ng
14:39:57     1000     16167   92,00    3,00    0,00    5,00   95,00     3  stress-ng
14:39:58     1000     16167  100,00    0,00    0,00    0,00  100,00     3  stress-ng
14:39:59     1000     16167   93,00    0,00    0,00    7,00   93,00     3  stress-ng
14:40:00     1000     16167   96,00    0,00    0,00    3,00   96,00     3  stress-ng
14:40:01     1000     16167   92,08    0,00    0,00    6,93   92,08     3  stress-ng
14:40:02     1000     16167   94,00    2,00    0,00    6,00   96,00     3  stress-ng
14:40:03     1000     16167   91,00    1,00    0,00    8,00   92,00     3  stress-ng
14:40:04     1000     16167   93,00    0,00    0,00    6,00   93,00     3  stress-ng
14:40:05     1000     16167   99,00    1,00    0,00    1,00  100,00     3  stress-ng
14:40:06     1000     16167   96,00    0,00    0,00    4,00   96,00     3  stress-ng
14:40:07     1000     16167   96,04    0,00    0,00    2,97   96,04     3  stress-ng
14:40:08     1000     16167   97,00    1,00    0,00    1,00   98,00     3  stress-ng
14:40:09     1000     16167   98,00    1,00    0,00    2,00   99,00     3  stress-ng
14:40:10     1000     16167   98,00    0,00    0,00    2,00   98,00     3  stress-ng
14:40:11     1000     16167   93,00    3,00    0,00    5,00   96,00     3  stress-ng

14:40:11      UID       PID    %usr %system  %guest   %wait    %CPU   CPU  Command
14:40:12     1000     16167  100,00    0,00    0,00    0,00  100,00     3  stress-ng
14:40:13     1000     16167   99,00    0,00    0,00    1,00   99,00     3  stress-ng
14:40:14     1000     16167   97,03    0,99    0,00    1,98   98,02     3  stress-ng
14:40:15     1000     16167   92,08    0,99    0,00    5,94   93,07     3  stress-ng
14:40:16     1000     16167   99,00    0,00    0,00    2,00   99,00     3  stress-ng
14:40:17     1000     16167   99,00    0,00    0,00    1,00   99,00     5  stress-ng
14:40:18     1000     16167   91,00    1,00    0,00    9,00   92,00     5  stress-ng
14:40:19     1000     16167   92,00    0,00    0,00    6,00   92,00     5  stress-ng
14:40:20     1000     16167   96,04    0,00    0,00    3,96   96,04     5  stress-ng
14:40:21     1000     16167   96,00    0,00    0,00    4,00   96,00     5  stress-ng
14:40:22     1000     16167   99,00    0,00    0,00    2,00   99,00     5  stress-ng
14:40:23     1000     16167   96,00    0,00    0,00    4,00   96,00     5  stress-ng
14:40:24     1000     16167   93,00    1,00    0,00    6,00   94,00     5  stress-ng
14:40:25     1000     16167   88,24    2,94    0,00    9,80   91,18     5  stress-ng
14:40:26     1000     16167   93,88    1,02    0,00    5,10   94,90     5  stress-ng
14:40:27     1000     16167   93,00    1,00    0,00    5,00   94,00     5  stress-ng
14:40:28     1000     16167   93,14    0,00    0,00    5,88   93,14     5  stress-ng
14:40:29     1000     16167   99,00    0,00    0,00    1,00   99,00     5  stress-ng
14:40:30     1000     16167   97,03    0,00    0,00    2,97   97,03     5  stress-ng
14:40:31     1000     16167   97,00    1,00    0,00    2,00   98,00     5  stress-ng
14:40:32     1000     16167   97,00    2,00    0,00    2,00   99,00     5  stress-ng
14:40:33     1000     16167   95,00    0,00    0,00    5,00   95,00     5  stress-ng
14:40:34     1000     16167   93,00    3,00    0,00    4,00   96,00     5  stress-ng
14:40:35     1000     16167   95,00    0,00    0,00    5,00   95,00     5  stress-ng
14:40:36     1000     16167   95,00    0,00    0,00    5,00   95,00     5  stress-ng
14:40:37     1000     16167   95,00    1,00    0,00    4,00   96,00     5  stress-ng
14:40:38     1000     16167   91,00    2,00    0,00    7,00   93,00     5  stress-ng
14:40:39     1000     16167   98,00    0,00    0,00    2,00   98,00     5  stress-ng
14:40:40     1000     16167   93,00    3,00    0,00    4,00   96,00     5  stress-ng
14:40:41     1000     16167   96,00    2,00    0,00    1,00   98,00     5  stress-ng
14:40:42     1000     16167   97,09    0,00    0,00    2,91   97,09     0  stress-ng
14:40:43     1000     16167   97,00    1,00    0,00    3,00   98,00     0  stress-ng
14:40:44     1000     16167   94,00    0,00    0,00    6,00   94,00     0  stress-ng
14:40:45     1000     16167   95,00    0,00    0,00    0,00   95,00     0  stress-ng
14:40:46     1000     16167   96,00    4,00    0,00    5,00  100,00     0  stress-ng
14:40:47     1000     16167   94,00    1,00    0,00    5,00   95,00     0  stress-ng
14:40:48     1000     16167   97,00    3,00    0,00    1,00  100,00     0  stress-ng
14:40:49     1000     16167   99,00    0,00    0,00    0,00   99,00     0  stress-ng
14:40:50     1000     16167   97,00    0,00    0,00    2,00   97,00     0  stress-ng
14:40:51     1000     16167   96,00    1,00    0,00    4,00   97,00     4  stress-ng
14:40:52     1000     16167   90,10    1,98    0,00    7,92   92,08     4  stress-ng

14:40:52      UID       PID    %usr %system  %guest   %wait    %CPU   CPU  Command
14:40:53     1000     16167   97,00    0,00    0,00    2,00   97,00     4  stress-ng
14:40:54     1000     16167   97,00    1,00    0,00    1,00   98,00     4  stress-ng
14:40:55     1000     16167   95,00    1,00    0,00    5,00   96,00     4  stress-ng
14:40:57     1000     16167   99,00    0,00    0,00    2,00   99,00     4  stress-ng
14:40:57     1000     16167   94,00    1,00    0,00    4,00   95,00     4  stress-ng
14:40:58     1000     16167   97,00    0,00    0,00    3,00   97,00     4  stress-ng
14:40:59     1000     16167   94,00    0,00    0,00    6,00   94,00     4  stress-ng
14:41:00     1000     16167   88,00    1,00    0,00   12,00   89,00     4  stress-ng
14:41:01     1000     16167   93,00    2,00    0,00    4,00   95,00     4  stress-ng
14:41:03     1000     16167   98,00    1,00    0,00    2,00   99,00     1  stress-ng
14:41:04     1000     16167   93,00    0,00    0,00    6,00   93,00     1  stress-ng
14:41:05     1000     16167   96,00    1,00    0,00    4,00   97,00     1  stress-ng
14:41:06     1000     16167   94,00    0,00    0,00    5,00   94,00     1  stress-ng
14:41:07     1000     16167   96,04    0,00    0,00    3,96   96,04     1  stress-ng
14:41:08     1000     16167   92,00    4,00    0,00    4,00   96,00     1  stress-ng
14:41:09     1000     16167   94,00    1,00    0,00    5,00   95,00     1  stress-ng
14:41:10     1000     16167   99,00    0,00    0,00    1,00   99,00     1  stress-ng
14:41:11     1000     16167   95,00    1,00    0,00    4,00   96,00     5  stress-ng
14:41:12     1000     16167   96,08    0,00    0,00    2,94   96,08     5  stress-ng
14:41:13     1000     16167   98,98    0,00    0,00    2,04   98,98     5  stress-ng
14:41:14     1000     16167   99,00    0,00    0,00    1,00   99,00     5  stress-ng
14:41:15     1000     16167   91,00    1,00    0,00    9,00   92,00     5  stress-ng
14:41:16     1000     16167   94,00    0,00    0,00    5,00   94,00     5  stress-ng
14:41:17     1000     16167   96,00    1,00    0,00    4,00   97,00     5  stress-ng
14:41:18     1000     16167   96,00    1,00    0,00    2,00   97,00     6  stress-ng
14:41:19     1000     16167   82,00    6,00    0,00   13,00   88,00     1  stress-ng
14:41:20     1000     16167   89,00    2,00    0,00    9,00   91,00     1  stress-ng
\end{verbatim}
Построим график потребление программой CPU:\\
\includegraphics[width=\textwidth]{image/cpu_usage.png}
Применяем все настройки, как для \textit{fft}. Ставим приоритет 19, пробуем разные политики планирования (самой выгодной опять оказалась NORMAL), закреплем процессы за конкретными процессорами, отключаем графический интерфейс.\\
\includegraphics[width=\textwidth]{image/cpu-cdouble-max.png}
Итого с \textbf{1'844'831} bogo ops мы повысили производительность до \textbf{3'428'179} bogo ops.\\
\includegraphics[width=\textwidth]{image/cpu_usage_max.png}
\end{document}
